\section{Diskussion}
\label{sec:Diskussion}
Die errechneten Werte sind in Tabelle \ref{tab:Erg} aufgeführt.
\begin{table}
  \centering
  \begin{tabular}{c|c c}
    \toprule
	Zu messende Größe & Zink & Kupfer \\
    \midrule
	Dicke des Plättchens & (\num{4.52 +- 0.08}) µm & (\num{6.19 +- 0.02}) µm \\
	Ladungsträgerdichte & (\num{1.76 +- 0.01}) $10^{28}$ $\frac{\text{C}}{\text{m}^3}$ & (\num{1.00 +- 0.04}) $10^{30}$ $\frac{\text{C}}{\text{m}^3}$\\
	Mittlere Flugzeit Kupfer & 3.91 \cdot $10^{-15}$ s & 6.71 \cdot $10^{-14}$ s \\
	Mittlere Geschwindigkeit & 6.19 µm / s & 0.354 mm / s \\
	Totalgeschwindigkeit & 3.59 \cdot $10^7$ m / s & 9.31 \cdot $10^6$ m / s\\
	mittlere freie Weglänge & 1.40 \cdot $10^{-8}$ m / s& 6.25 \cdot $10^{-8}$ m / s \\
	Beweglichkeit der Elektronen & 3.4 \cdot $10^{-4}$ C \cdot s / kg & 5.9 \cdot $10^{-3}$ C \cdot s / kg \\
	Ladungsträger pro Atom & 11.7 & 0.27 \\

    \bottomrule
  \end{tabular}
  \caption{Berechneten Werte für Kupfer und Zink}
  \label{tab:Erg}
\end{table}
Sie können nicht direkt mit Literaturwerten verglichen werden, da keine geeignete Quelle gefunden wurde. Dennoch ist anzuzweifeln das Zink 11.7 Ladungsträger pro Atom besitzt. 
