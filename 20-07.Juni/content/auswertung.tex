\section{Auswertung}
\label{sec:Auswertung}
\subsection{Hystereskurve}
Die gegen den Spulenstrom aufgetragene Feldstärke ist in dem Diagramm \ref{fig:Hyst} aufgetragen.
\begin{figure}
  \centering
  \includegraphics[height=8cm]{Hysterese.pdf}
  \caption{Hysteresekurve}
  \label{fig:Hyst}
\end{figure}
Es kann aus dem Diagramm eine Magnetisierung des Eisenkerns und die existens des B-Feld abgelesen werde.

\subsection{Mikroskopische Leitfähigkeitsparameter}
Zunächst wird die Dicke des vom B-Feld durchsetzten Plättchens berechent. Dies geschieht indem zunächst aus der in Tabelle \ref{tab:RZK} aufgetragenene Stromstärke und Spannung der Widerstand des Verwendeten Materials berechnet wird.
\begin{table}
  \centering
  \begin{tabular}{c|c c|c c}
    \toprule
    $I$ / A & $U_\text{Zink}$ / mV & $R_\text{Zink}$ / $m\Omega$ & $U_\text{Kupfer}$ / mV & $R_\text{Kupfer}$ / $m\Omega$ \\
    \midrule
	0.5	& 5	& 100	& ---	& ---	\\
	1	& 10	& 100	& 5	& 50	\\
	1.5	& 15	& 100	& ---	& ---	\\
	2	& 19	& 95	& 10	& 50	\\
	2.5	& 25	& 100	& ---	& ---	\\
	3	& 30	& 100	& 16 	& 53	\\
	3.5	& 34	& 97	& ---	& ---	\\
	4	& 39	& 98	& 23	& 58	\\
	4.5	& 44	& 98	& ---	& ---	\\
	5	& 49	& 98	& 28	& 56	\\
	5.5	& 55	& 100	& ---	& ---	\\
	6	& 59	& 98	& 34	& 57	\\
	6.5	& 64	& 98	& ---	& ---	\\
	7	& 69	& 99	& 40  	& 57	\\
	7.5	& 74	& 99	& ---	& ---	\\
	8	& 79	& 99	& 46	& 57	\\
	9	&	&	& 52	& 58	\\
	10	&	&	& 58	& 58	\\
    \midrule
    \multicolumn{2}{c}{gemitteleter Widerstand:}& (\num{98.6 +- 0.03}) & --- & (\num{55 +- 1}) \\
    \bottomrule
  \end{tabular}
  \caption{Wiederstände von Zink und Kupfer}
  \label{tab:RZK}
\end{table}
Anhand des berechneten Widerstand lässt sich nach Gleichung \ref{eqn:R} die Dicke des Plättchens berechnen. Es ergibt sich für Zink eine Dicke von
\begin{equation}
  d_\text{Zink} = (\num{4.52 +- 0.08}) \,  \text{µm}
  \label{eqn:dZ}
\end{equation}
und für Kupfer
\begin{equation}
  d_\text{Kupfer} = (\num{6.19 +- 0.02}) \, \text{µm} \. .
  \label{eqn:dK}
\end{equation}
Über die Gemessene Hallspannung wird wird die Ladungsträgerdichte berechnet und in Tabelle \ref{tab:n} aufgeführt.
\begin{table}
  \centering
  \begin{tabular}{c|c c c c}
    \toprule
    $I$ \ A & $U_\text{Kupfer}$ / V & $n_\text{Kupfer}$ / $ \frac{10^{30} \text{C}}{\text{m}^3} $ & $U_\text{Zink}$ / V & $n_\text{Zink}$ / $\frac{10^{28}\text{C}}{\text{m}^3}$ \\
    \midrule
    0.5	& 0.000 & --- 	& 0.036	& 1.72 \\
    1	& 0.001 & 1.65	& 0.070	& 1.77 \\
    1.5	& 0.002 & 1.23	& 0.102	& 1.78 \\
    2	& 0.003 & 1.10	& 0.135	& 1.78 \\
    2.5	& 0.004 & 1.03	& 0.169	& 1.78 \\
    3	& 0.005 & 0.99	& 0.203	& 1.78 \\
    3.5	& 0.006 & 0.96	& 0.234	& 1.70 \\
    4	& 0.007 & 0.94	& 0.270	& 1.78 \\
    4.5	& 0.008 & 0.93	& 0.306	& 1.77 \\
    5	& 0.009 & 0.92	& 0.340	& 1.77 \\
    5.5	& 0.010 & 0.91	& 0.373	& 1.77 \\
    6	& 0.011 & 0.90	& 0.410	& 1.76 \\
    6.5	& 0.011 & 0.97	& 0.445	& 1.76 \\
    7	& 0.012 & 0.96	& 0.478	& 1.76 \\
    7.5	& 0.013 & 0.95	& 0.515	& 1.75 \\
    8	& 0.014 & 0.94	& 0.550	& 1.75 \\
    8.5	& 0.015 & 0.93	& ---	& ---  \\
    9	& 0.016 & 0.93	& ---	& ---  \\
    9.5	& 0.017 & 0.92	& ---	& ---  \\
    10	& 0.017 & 0.97	& ---	& ---  \\
    \midrule
    \multicolumn{2}{c}{gemittelete Ladungsträgerdichte:}& (\num{1.00 +- 0.04}) & --- & (\num{1.76 +- 0.01}) \\
    \bottomrule
  \end{tabular}
  \caption{Ladungsträgerdichte von Zink und Kupfer}
  \label{tab:n}
\end{table}
Aus der Ladungsdichte wird nach Gleichung \ref{eqn:rho} die Mittlere Flugzeit für Zink
\begin{equation}
  \overline{\tau_\text{Zink}} = 6.71 \cdot 10^{-14} \ \text{s}
  \label{eqn:tZ}
\end{equation}
sowie Kupfer
\begin{equation}
  \overline{\tau_\text{Kupfer}} = 3.91 \cdot 10^{-15} \ \text{s}
    \label{eqn:tK}
\end{equation}
berechnet. Die mittlere Driftgeschwindigkeit berechnet sich aus Gleichung \ref{eqn:drift} zu
\begin{eqnarray}
  \overline{v_\text{Kupfer}} = 6.19 \frac{\text{µm}}{\text{s}} \ , \\
  \overline{v_\text{Zink}} = 0.354 \frac{\text{mm}}{\text{s}} \ .
\end{eqnarray}
und die Totalgeschwindigkeit $v$ nach Gleichung \ref{eqn:} zu
\begin{eqnarray}
  v_\text{Kupfer} = 3.59 \cdot 10^7 \frac{\text{m}}{\text{s}} \ , \\
  v_\text{Zink} = 9.31 \cdot 10^6 \frac{\text{m}}{\text{s}} \ .
\end{eqnarray}
Aus dem Produkt der mittleren Fugzeit und der Totalgeschwindigkeit ergibt sich die mittlere freie Weglänge von
\begin{eqnarray}
  \overline{l_\text{Kupfer}} = 1.40 \cdot 10^{-8} \, \text{m} \ \text{und} \\
  \overline{l_\text{Zink}} = 6.25 \cdot 10^{-8} \, \text{m} \ .\\
\end{eqnarray}
Die Beweglichkeit $µ$ der Elektronen wird aus Gleichung \ref{eqn:Beweglichkeit} berechnet und beträgt für Kupfer
\begin{equation}
  µ_\text{Kupfer} = 3.4 \cdot 10^{-4} \, \frac{\text{C s}}{\text{kg}}
\end{equation}
und für Zink
\begin{equation}
  µ_\text{Zink} = 5.9 \cdot 10^{-3} \, \frac{\text{C s}}{\text{kg}}
\end{equation}
Für die Anzahl wird zunächst die Anzahl der Atomdichte entsprechend Formel \ref{eqn:} berechnet und diese durch die Ladungsträgerdicht geteilt. Für Kupfer ergibt sich eine Atomdichte von
\begin{equation}
  \frac{8920 \frac{kg}{m^3}}{63.4 u} = 8.5 \cdot 10^{28} \, \frac{1}{\text{m}^3}
\end{equation}
woraus aus dem Verhältniss Atomdichte durch Ladungsdichte sich eine Anzahl von 11.76 Ladungsträger pro Atom ergeben. Für Zink ergibt sich aus analoger Rechnung 0.27 Ladungsträger pro Atom.
