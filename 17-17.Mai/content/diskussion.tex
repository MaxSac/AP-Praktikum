\section{Diskussion}
\label{sec:Diskussion}
Die Gitterkonstante $g$ beträgt
\begin{align*}
  g = (\num{9.81 +- 0.04}) \cdot 10^{-7}\ \text{m} \ .
\end{align*}
Die Eichgröße $\Psi$ beträgt
\begin{align*}
  \Psi = (\num{9.827 +- 0.008}) \cdot 10^{-7}\ \frac{\text{m}}{\text{Skt}} \ .
\end{align*}
Die inneren Abschirmungszahlen $\sigma_2$ für die Alkali-Atome betragen
\begin{align*}
  \sigma_{2,\text{Na}} & = \num{10.82 +- 0.01} \\
  \sigma_{2,\text{K}}  & = \num{18.808 +- 0.008} \\
  \sigma_{2,\text{Rb}} & = 36.65
\end{align*}
Die ausgerechneten Werte für die Abschirmungszahlen liegen im erwarteten Bereich und haben im Mittel nur einen sehr kleinen Fehler. 
