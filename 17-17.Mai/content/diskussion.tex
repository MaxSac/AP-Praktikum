\section{Diskussion}
\label{sec:Diskussion}
Die Gitterkonstante $g$ beträgt
\begin{align*}
  g = (\num{9.81 +- 0.04}) \cdot 10^{-7}\ \text{m} \ .
\end{align*}
Die Eichgröße $\Psi$ beträgt
\begin{align*}
  \Psi = (\num{3.198 +- 0.007}) \cdot 10^{-7}\ \frac{\text{m}}{\text{Skt}} \ .
\end{align*}
Die inneren Abschirmungszahlen $\sigma_2$ für die Alkali-Atome betragen
\begin{align*}
  \sigma_{2,\text{Na}} & = \num{7.8 +- 0.4} \\
  \sigma_{2,\text{K}}  & = \num{12.6 +- 0.3} \\
  \sigma_{2,\text{Rb}} & = 27.5
\end{align*}
Die ausgerechneten Werte für die Abschirmungszahlen liegen im erwarteten Bereich und haben im Mittel nur einen kleinen Fehler. \\
Die Fehler der ausgerechneten Werte entstehen vornehmlich durch Ablesefehler bei der Winkelbestimmung.
