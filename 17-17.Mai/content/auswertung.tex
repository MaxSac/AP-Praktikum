\section{Auswertung}
\label{sec:Auswertung}

\subsection{Berechnung der Gitterkonstanten $g$ aus dem Helium-Spektrum}
Die zu den gegebenen Wellenlängen $\lambda$ gemessenen Beugungswinkel $\phi$, für Helium, sind in Tabelle \eqref{tab:Mess1} aufgelistet.

\begin{table}[H] %Beugungswinkel, Wellenlänge
  \centering
  \begin{tabular}{c| c c c}
    \toprule
    Farbe & $\lambda$ / $10^{-9}$\, m & $\phi$ / rad & $\sin(\phi)$ \\
    \midrule
      violett   & 438.8 & 0.456 & 0.440 \\
      violett   & 447.1 & 0.466 & 0.449 \\
      blau      & 471.3 & 0.492 & 0.473 \\
      blaugrün  & 492.2 & 0.517 & 0.494 \\
      grün      & 501.6 & 0.531 & 0.506 \\
      grün      & 504.8 & 0.534 & 0.509 \\
      gelb      & 587.6 & 0.635 & 0.593 \\
      rot       & 667.8 & 0.738 & 0.673 \\
      dunkelrot & 706.5 & 0.794 & 0.713 \\
    \bottomrule
  \end{tabular}
  \caption{Die Wellenlänge und der zugehörige Beugungswinkel aus dem Helium-Spektrum.}
  \label{tab:Mess1}
\end{table}

Die Gitterkonstante $g$ wird mit Hilfe einer linearen Regression der Gleichung \eqref{eqn:phi} bestimmt, indem $\sin(\phi)$ gegen $\lambda$ aufgetragen wird (siehe Diagramm \eqref{fig:Gitterkonstante}). Aus der Regression ergeben sich folgende Werte:
\begin{align*}
  k & = 1 \\
  b & = (\num{-6 +- 2}) \cdot 10^{-3} \\
  \Rightarrow g & = (\num{9.81 +- 0.04}) \cdot 10^{-7}\ \text{m}
\end{align*}

\begin{figure}[H]
  \centering
  \includegraphics[height=8cm]{build/Gitterkonstante}
  \caption{Ausgleichsgerade zur Berechnung der Gitterkonstante.}
  \label{fig:Gitterkonstante}
\end{figure}

\subsection{Bestimmung der Eichgröße}
Die Eichgröße $\Psi$ lässt sich mit
\begin{align}
  \Psi = \frac{\lambda_1 - \lambda_2}{\cos(\overline{\phi}_{1, 2}) \Delta t}
\end{align}
berechnen. Die dafür benötigten Werte sind in den Tabellen \eqref{tab:Mess2} und \eqref{tab:Mess2.1} angegeben.

\begin{table}[H]
  \centering
  \begin{tabular}{c|c|c|c|c}
    & $\phi_1$ / rad & $\phi_2$ / rad & $\overline{\phi}_{1,2}$ / rad & $\cos(\overline{\phi}_{1,2})$ \\
    \hline
    Helium                  & 0.531 & 0.534 & $\num{0.532 +- 0.002}$ &
                            $\num{0.862 +- 0.001}$ \\
    \hline
    \multirow{3}*{Natrium}  & 0.606 & 0.609 & $\num{0.607 +- 0.002}$ &
                            $\num{0.821 +- 0.001}$ \\
                            & 0.632 & 0.637 & $\num{0.634 +- 0.003}$ & $\num{0.805 +- 0.002}$ \\
                            & 0.667 & 0.670 & $\num{0.668 +- 0.002}$ & $\num{0.785 +- 0.001}$ \\
    \hline
    \multirow{4}*{Kalium}   & 0.532 & 0.534 & $\num{0.533 +- 0.001}$ &
                            $\num{0.8612 +- 0.0004}$ \\
                            & 0.534 & 0.536 & $\num{0.535 +- 0.001}$ &
                            $\num{0.8603 +- 0.0004}$ \\
                            & 0.620 & 0.621 & $\num{0.620 +- 0.001}$ &
                            $\num{0.814 +- 0.001}$ \\
                            & 0.623 & 0.625 & $\num{0.624 +- 0.001}$ &
                            $\num{0.812 +- 0.001}$ \\
    \hline
    Rubidium                & 0.675 & 0.689 & $\num{0.682 +- 0.007}$ &
                            $\num{0.776 +- 0.004}$ \\
    \hline
  \end{tabular}
  \caption{Messwerte zur Bestimmung der Eichgröße Teil 1}
  \label{tab:Mess2}
\end{table}

\begin{table}[H]
  \centering
  \begin{tabular}{c|c|c|c|c}
    & $\lambda_1$ / $10^{-9}$ m & $\lambda_2$ / $10^{-9}$ m & $\Delta t$ / Skt & $\Psi$ / $10^{-7} \cdot \frac{\text{m}}{\text{Skt}}$ \\
    \hline
    Helium                  & 501.6 & 504.8 & 1.25 & $\num{9.83 +- 0.01}$
                            \\
    \hline
    \multirow{3}*{Natrium}  & 559.4 & 562.2 & 0.24 & $\num{9.83 +- 0.01}$
                            \\
                            & 580.3 & 584.5 & 0.18 & $\num{9.83 +- 0.02}$ \\
                            & 607.6 & 610.3 & 0.08 & $\num{9.83 +- 0.01}$ \\
    \hline
    \multirow{4}*{Kalium}   & 498.7 & 500.2 & 0.89 & $\num{9.826 +- 0.005}$
                            \\
                            & 500.2 & 501.6 & 0.67 & $\num{9.826 +- 0.005}$ \\
                            & 570.6 & 572.0 & 0.86 & $\num{9.826 +- 0.006}$ \\
                            & 573.4 & 574.8 & 0.81 & $\num{9.826 +- 0.006}$ \\
    \hline
    Rubidium                & 614.3 & 625.0 & 2.92 & $\num{9.83 +- 0.06}$
                            \\
    \hline
  \end{tabular}
  \caption{Messwerte zur Bestimmung der Eichgröße Teil 2}
  \label{tab:Mess2.1}
\end{table}

Der Mittelwert der Eichgrößen beträgt:
\begin{align*}
  \Psi = (\num{9.827 +- 0.008}) \cdot 10^{-7} \frac{\text{m}}{\text{Skt}}
\end{align*}

\subsection{Berechnung der Abschirmungszahl}
Für die Berechnung der inneren Abschirmungszahl $\sigma_2$ werden mehrere Schritte benötigt. Zu Beginn muss die Wellenlänge der Dublettlinien mit Hilfe der gemessenen Beugungswinkel und der bereits bestimmten Gitterkonstante nach Gleichung \eqref{eqn:phi} bestimmt werden. Dann wird mit Gleichung \eqref{eqn:ed} die Energiedifferenz der Dublettlinien berechnet. Zu letzt soll die Gleichung \eqref{eqn:ded} nach $\sigma_2$ umgestellt werden
\begin{equation*}
  \sigma_2 = z - \sqrt[4]{\Delta E_\text{D} \cdot l(l + 1)\, \frac{n^3}{R_\infty \alpha^2}} \ .
\end{equation*}
Dabei ist bei Natrium $n = 3$, bei Kalium $n = 4$ und bei Rubidium $n = 5$ und für alle drei Alkali-Atome ist $l = 1$. Die Ergebnisse und Zwischenergebnisse befinden sich in Tabelle \eqref{tab:Mess3}.

\begin{table}[H]
  \centering
  \begin{tabular}{c|c|c|c}
    \hline
    $\phi$ / rad & $\lambda$ / $10^{-9}$ m & $\Delta E_\text{D}$ / eV & $\sigma_2$ \\
    \hline
    \multicolumn{4}{c}{Natrium} \\
    \hline
    0.606 & 559.4 & \multirow{2}{*}{0.011} & \multirow{2}{*}{10.82} \\
    0.609 & 562.2 & & \\
    \hline
    0.632 & 580.3 & \multirow{2}{*}{0.015} & \multirow{2}{*}{10.81} \\
    0.637 & 584.5 & & \\
    \hline
    0.667 & 607.6 & \multirow{2}{*}{0.009} & \multirow{2}{*}{10.83} \\
    0.670 & 610.3 & & \\
    \hline
    \multicolumn{3}{c}{Mittelwert für $\sigma_2$ =} & $\num{10.82 +- 0.01}$ \\
    \hline
    \multicolumn{4}{c}{Kalium} \\
    \hline
    0.532 & 498.7 & \multirow{2}{*}{0.007} & \multirow{2}{*}{18.80} \\
    0.534 & 500.2 & & \\
    \hline
    0.534 & 500.2 & \multirow{2}{*}{0.008} & \multirow{2}{*}{18.80} \\
    0.536 & 501.6 & & \\
    \hline
    0.620 & 570.6 & \multirow{2}{*}{0.005} & \multirow{2}{*}{18.82} \\
    0.621 & 572.0 & & \\
    \hline
    0.623 & 573.4 & \multirow{2}{*}{0.005} & \multirow{2}{*}{18.82} \\
    0.625 & 574.8 & & \\
    \hline
    \multicolumn{3}{c}{Mittelwert für $\sigma_2$ =} & $\num{18.808 +- 0.008}$ \\
    \hline
    \multicolumn{4}{c}{Rubidium} \\
    \hline
    0.675 & 614.3 & \multirow{2}{*}{0.034} & \multirow{2}{*}{36.65} \\
    0.689 & 625.0 & & \\
    \hline
  \end{tabular}
  \caption{Messwerte für die Bestimmung der inneren Abschirmungszahlen.}
  \label{tab:Mess3}
\end{table}

\subsection{Bestimmung der Distanz zwischen den Dublettlinien}
Die Distanz der Dublettlinien ist in folgender Tabelle aufgeführt.

\begin{table}[H]
  \centering
  \begin{tabular}{c c}
    \toprule
    & $\Delta s$ / $10^{-3}$ m \\
    \hline
    \multirow{3}*{Natrium} & 0.19 \\
                           & 0.43 \\
                           & 0.58 \\
    \hline
    \multirow{3}*{Kalium}  & 2.06 \\
                           & 1.94 \\
                           & 2.14 \\
                           & 1.61 \\
    \hline
    Rubidium               & 7.01 \\
    \hline
  \end{tabular}
  \caption{Die Distanz der Dublettlinien.}
  \label{tab:}
\end{table}
