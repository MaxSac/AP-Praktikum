\section{Diskussion}
\label{sec:Diskussion}
Die Standardabweichung der Steigung und der daraus resultierende Fehler für $\epsilon$, lassen sich durch die Ansprechzeit der Thermosäule nach Moll und der ungenauen Temperaturmessung erklären. Außerdem sind große Schwankungen in der Thermospannung zu erkennen wenn Personen an dem Versuchaufbau vorbei laufen.
\\
Ein Vergleich mit Literaturwerten ist hier nicht Sinnvoll da wir von der idealisierten Annahme ausgehen, dass die schwarze Oberfläche ein Emissionsvermögen von $\epsilon = 1$ besitzt und alle anderen Werte von diesem abgeleitet werden.
\\
Bei der Thermospannung im Verhältnis zum Abstand ist deutlich ein linearer Zusammenhang zwischen $U \propto A$ zu erkennen.
