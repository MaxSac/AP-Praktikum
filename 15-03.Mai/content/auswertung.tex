\section{Auswertung}
\label{sec:Auswertung}
Die zu Beginn aufgenommenen Daten zu den zwei Hochvakuumdioden werden im Folgenden aufgelistet:\\
\textbf{Diode 1:} \\
  Kathodenoberfläche: f = 0.32 cm$^{2}$ , \\
  Maximaler Heizstrom: I = 2.5 A \\
\textbf{Diode 2:} \\
  Kathodenoberfläche: f = 0.35 cm$^{2}$ , \\
  Maximaler Heizstrom: I = 2.0 A \\

  \begin{table}[H] %Messdaten
    \small
    \centering
    \begin{tabular}{c||c|c|c|c|c}
      \toprule
      Beschleunigungs- & \multicolumn{2}{c|}{Diode 1} & \multicolumn{3}{c}{Diode 2} \\
      spannung & $I_\text{heiz}$ = 2.5 A & $I_\text{heiz}$ = 2.4 A & $I_\text{heiz}$ = 2.0 A & $I_\text{heiz}$ = 1.9 A & $I_\text{heiz}$ = 1.8 A \\
      \midrule
      U / V & $I_1$ / mA & $I_2$ / mA & $I_3$ / mA & $I_4$ / mA & $I_5$ / mA \\
      \midrule
      0	  & 0.000 & 0.000 & 0.000	& 0.000	& 0.000 \\
      1	  &       &       & 0.000	& 0.000	& 0.000 \\
      2	  &       &       & 0.000	& 0.000	& 0.000 \\
      3	  &       &       & 0.001	& 0.000	& 0.000 \\
      4	  &       &       & 0.002	& 0.001	& 0.000 \\
      5	  & 0.015 & 0.017 & 0.006	& 0.002	& 0.000 \\
      6	  &       &       & 0.007	& 0.004	& 0.000 \\
      7	  &       &       & 0.015	& 0.006	& 0.000 \\
      8	  &       &       & 0.015	& 0.009	& 0.000 \\
      9	  &       &       & 0.022	& 0.013	& 0.001 \\
      10	& 0.035 & 0.037 & 0.033	& 0.018	& 0.004 \\
      11	&       &       & 0.040	& 0.021	& 0.008 \\
      12	&       &       & 0.046	& 0.025	& 0.011 \\
      13	&       &       & 0.050	& 0.028	& 0.012 \\
      14	&       &       & 0.060	& 0.030	& 0.015 \\
      15	& 0.055 & 0.062 & 0.066	& 0.034	& 0.015 \\
      16	&       &       & 0.074	& 0.037	& 0.017 \\
      17	&       &       & 0.074	& 0.040	& 0.019 \\
      18	&       &       & 0.086	& 0.043	& 0.019 \\
      19	&       &       & 0.098	& 0.045	& 0.020 \\
      20	& 0.083 & 0.087 & 0.102	& 0.046	& 0.021 \\
      21	&       &       & 0.109	& 0.048	& 0.021 \\
      22	&       &       & 0.113	& 0.051	& 0.022 \\
      23	&       &       & 0.117	& 0.052	& 0.022 \\
      24	&       &       & 0.123	& 0.053	& 0.022 \\
      25	& 0.113 & 0.117 & 0.122	& 0.053	& 0.023 \\
      26	&       &       & 0.130	& 0.054	& 0.023 \\
      27	&       &       & 0.130	& 0.055	& 0.023 \\
      28	&       &       & 0.133	& 0.056	& 0.023 \\
      29	&       &       & 0.132	& 0.056	& 0.023 \\
      30	& 0.148 & 0.150 & 0.134	& 0.056	& 0.023 \\
      35	& 0.187 & 0.186 & 0.140	& 0.057	& 0.024 \\
      40	& 0.219 & 0.223 & 0.144	& 0.058	& 0.024 \\
      45	& 0.266 & 0.260 & 0.147	& 0.059	& 0.025 \\
      50	& 0.315 & 0.302 & 0.148	& 0.060	& 0.025 \\
      55	& 0.364 & 0.344 & 0.149	& 0.060	& 0.025 \\
      60	& 0.430 & 0.382 & 0.150	& 0.061	& 0.025 \\
      70	& 0.532 & 0.510 & 0.153	& 0.062	& 0.025 \\
      80	& 0.635 & 0.612 & 0.154	& 0.062	& 0.025 \\
      90	& 0.750 & 0.718 & 0.156	& 0.063	& 0.026 \\
      100	& 0.875 & 0.821 & 0.156	& 0.063	& 0.026 \\
      110	& 1.020 & 0.933 & 0.155	& 0.063	& 0.026 \\
      120	& 1.184 & 1.053 & 0.157	& 0.064	& 0.026 \\
      130	& 1.330 & 1.168 & 0.157	& 0.064	& 0.026 \\
      140	& 1.470 & 1.296 & 0.158	& 0.064	& 0.027 \\
      150	& 1.622 & 1.462 & 0.159	& 0.064	& 0.027 \\
      170	& 1.929 & 1.660 & 0.160	& 0.065	& 0.027 \\
      190	& 2.230 & 1.899 & 0.161	& 0.065	& 0.027 \\
      210	& 2.550 & 2.060 & 0.163	& 0.065	& 0.027 \\
      230	& 2.860 & 2.270 & 0.164	& 0.066	& 0.027 \\
      250	& 3.070 & 2.480 & 0.165	& 0.066	& 0.028 \\
      \bottomrule
    \end{tabular}
    \caption{Messdaten für den ersten Versuchsteil.}
    \label{tab:Messdaten1}
  \end{table}

\subsection{Kennlinienschar zu den Dioden und deren Sättigungsstrom}
\label{sec:K}
Die Messdaten aus Tabelle (\ref{tab:Messdaten1}) sind in den Abbildungen (\ref{fig:Kenn1}) und (\ref{fig:Kenn2}) aufgetragen.

\begin{figure}[H] %Kennlinienschar Diode 1
  \centering
  \includegraphics[height=7cm]{build/Kenn1}
  \caption{Kennlinienschar für die 1. Diode}
  \label{fig:Kenn1}
\end{figure}

\begin{figure}[H] %Kennlinienschar Diode 2
  \centering
  \includegraphics[height=7cm]{build/Kenn2}
  \caption{Kennlinienschar für die 2. Diode}
  \label{fig:Kenn2}
\end{figure}

\newpage

Der Sättigungsstrom $I_\text{s}$ für die beiden Dioden lässt sich, aus den Abbildungen (\ref{fig:Kenn1}) und (\ref{fig:Kenn2}), zu
\begin{align*}
  I_{\text{s},1} & \approx 4.000 \, \text{mA} \\
  I_{\text{s},2} & \approx 3.800 \, \text{mA} \\
  I_{\text{s},3} & \approx 0.175 \, \text{mA} \\
  I_{\text{s},4} & \approx 0.075 \, \text{mA} \\
  I_{\text{s},5} & \approx 0.035 \, \text{mA}
\end{align*}
abschätzen.

\subsection{Gültigkeitsbereich des Langmuir-Schottkyschen Raumladungsgesetzes}
\label{sec:R}
Die Kennlinien $I_1$ und $I_2$ liegen komplett in dem Raumladungsbereich und der Exponent des Raumladungsgesetzes (\ref{eqn:jLS}) wird durch eine Regression zu
\begin{align*}
  \text{Exponent von}\ I_1 & =  \num{1.379 +- 0.001} \ , \\
  \text{Exponent von}\ I_2 & =  \num{1.345 +- 0.002}
\end{align*}
bestimmt.

\begin{figure}[H]
  \centering
  \includegraphics[height=7cm]{build/Raum}
  \caption{Die gefittete Kurve für das Raumladungsgebiet der ersten Diode}
  \label{fig:}
\end{figure}

Für den Fit wurde folgende Gleichung verwendet:
\begin{align*}
  j = \frac{4}{9}\, \varepsilon_0\, \sqrt{2\, e_0/m_0}\frac{V^x}{a^2} \\
  (x = \text{gesuchter Exponent})
\end{align*}

\subsection{Anlaufstromgebiet und Kathodentemperatur}
\label{sec:A}
Die aufgenommenen Messdaten und die korrigierten Spannungen sind in Tabelle (\ref{tab:Messdaten2}) aufgeführt. Dieser Versuchsteil ist nur für die erste Diode durchgeführt worden. Da die Spannungen im nA Bereich liegen und das Ampermeter einen Innenwiderstand von 1 M$\Omega$ besitzt müssen die Spannungen mit
\begin{align*}
  U_\text{korr} = U - R_\text{innen} \cdot I
\end{align*}
korrigiert werden.

\begin{table}[H] %Messdaten
  \centering
  \begin{tabular}{c | c | c}
    \toprule
    U / V & I / nA & $U_\text{korr}$ / V \\
    \midrule
    0.00 & 9.00 &-0.009 \\
    0.05 & 6.70 & 0.043 \\
    0.10 & 6.00 & 0.094 \\
    0.15 & 5.30 & 0.145 \\
    0.20 & 4.00 & 0.196 \\
    0.25 & 3.10 & 0.247 \\
    0.30 & 2.30 & 0.298 \\
    0.35 & 1.80 & 0.348 \\
    0.40 & 1.30 & 0.399 \\
    0.45 & 1.10 & 0.449 \\
    0.50 & 0.96 & 0.499 \\
    0.55 & 0.75 & 0.549 \\
    0.60 & 0.59 & 0.599 \\
    0.65 & 0.45 & 0.650 \\
    0.70 & 0.36 & 0.700 \\
    0.75 & 0.28 & 0.750 \\
    0.80 & 0.22 & 0.800 \\
    0.85 & 0.17 & 0.850 \\
    0.90 & 0.14 & 0.900 \\
    0.95 & 0.10 & 0.950 \\
    1.00 & 0.09 & 1.000 \\
    \bottomrule
  \end{tabular}
  \caption{Messwerte zum Anlaufstromgebiet}
  \label{tab:Messdaten2}
\end{table}

Nun wird mit Hilfe von der Gleichung (\ref{eqn:jv}) eine Ausgleichsrechnung durchgeführt, um die Kathodentemperatur $T$ zu bestimmen. Für die Temperatur ergibt sich folgender Wert:
\begin{align*}
  T = (\num{2700 +- 80}) \, \text{K}
\end{align*}
Für den Fit wurde folgende Gleichung verwendet:
\begin{align*}
  I = j \exp\left( - \frac{e_0 V}{k_\text{B} T} \right)
\end{align*}
Die Normierungskonstante $j$ ergibt sich zu:
\begin{align*}
  j = (\num{8.7 +- 0.2}) \cdot 10^{-9} \, \text{A}
\end{align*}
\begin{figure}[H]
  \centering
  \includegraphics[height=7cm]{build/Anlauf}
  \caption{Anlaufstromgebiet für die erste Diode}
  \label{fig:anlauf}
\end{figure}

\subsection{Kathodentemperatur aus der Leistungsbilanz des Heiztromfadens}
\label{sec:L}
Mit dem Heizstrom $I_\text{heiz}$ und der Heizspannung $U_\text{heiz}$ aus Tabelle (\ref{tab:heiz}) und der Gleichung (\ref{eqn:T}) lässt sich die Kathodentemperatur berechnen.
\begin{equation}
  T = \left(\frac{I_\text{heiz}\, U_\text{heiz} - N_\text{WL}}{f\, \eta\, \sigma}\right)^{\frac{1}{4}}
  \label{eqn:T}
\end{equation}
Wobei $N_\text{WL}$ zu 0.95 W geschätzt wird, $f$ die Kathodenoberfläche ist, $\eta = 0.28$ dem Emissionsgrad der Oberfläche und $\sigma = 5.7 \cdot 10^{-12} \, \frac{\text{W}}{\text{cm}^2\, \text{K}^4}$ der Stefan-Boltzmannschen Strahlungskonstanten entspricht.

\begin{table}[H]
  \centering
  \begin{tabular}{c c c}
    \toprule
    $I_\text{heiz}$ / A & $U_\text{heiz}$ / V & T / K \\
    \midrule
    2.5 & 6.2 & 2310 \\
    2.4 & 6.0 & 2265 \\
    2.0 & 5.0 & 2051 \\
    1.9 & 4.3 & 1951 \\
    1.8 & 4.0 & 1870 \\
    \bottomrule
  \end{tabular}
  \caption{Die Kathodentemperatur mit dem zugehörigem Heizstrom und der Heizspannung.}
  \label{tab:heiz}
\end{table}

\subsection{Austrittsarbeit für Wolfram}
Die Richardson-Gleichung (\ref{eqn:js}) lässt sich wie folgt umstellen:
\begin{equation}
  \Phi = - \frac{k\, T}{e_0} \ln \left(\frac{I_\text{s}\, h^3}{4\, \pi\, f\, e_0\, m_0\, k^2\, T^2}\right)
\end{equation}
Daraus lässt sich mit den Kathodentemperaturen aus Kapitel (\ref{sec:K}) und dem Sättigungsstrom $I_\text{s}$ die Austrittsarbeit von Wolfram bestimmen. In der folgenden Tabelle ist die Austrittsarbeit aufgelistet.

\begin{table}[H]
  \centering
  \begin{tabular}{c c c}
    \toprule
    T / K & $I_\text{s}$ / mA & $\Phi$ / eV \\
    \midrule
    2310 & 4.000 & 6.324 \\
    2265 & 3.800 & 6.204 \\
    2051 & 0.175 & 4.887 \\
    1951 & 0.075 & 4.775 \\
    1870 & 0.035 & 4.685 \\
    \bottomrule
  \end{tabular}
  \caption{Die Austrittsarbeit von Wolfram}
  \label{tab:}
\end{table}

Der Mittelwert der Austrittsarbeit liegt bei ($\num{4.78 +- 0.08}$) eV. Allerdings wurden die ersten beiden Werte für die Berechnung des Mittelwertes nicht betrachtet, da sie sehr weit vom Mittel abweichen.
