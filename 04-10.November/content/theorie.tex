\section{Theoretische Grundlage}
\label{sec:Theorie}
Im folgenden Versuch soll die Wärmeleitfähigkeit verschiedener Stoffe bestimmt werden. Zur Wärmeleitung kommt es wenn ein Temperaturgefälle ($\partial T / \partial x$) auf dem Stab exsistiert. Dabei wird durch einen Querschnittsfläche $A$, in der Zeit $dt$ die Wärmemenge $dQ$ transportiert.
\begin{equation}
	\label{eqn:Q/t}
	dQ = - \kappa A \frac{\partial T}{\partial x} dt
\end{equation}
$\kappa$ ist eine materialäbhängige Konstante und dient als Proportionalitätsfaktor. In Analogie zur Elektrodynamik definiert man sich eine Wärmestromdichte 
\begin{equation}
	j_w = - \kappa \frac{\partial T}{\partial x}
\end{equation}
die den Verschiebungsstrom zwischen den Temperaturdifferenzen wiedergibt. Damit lässt sich eine Kontinuitätsgleichung formulieren
\begin{equation}
	\label{eqn:KG}
	\frac{\partial T}{\partial t} = \frac{\kappa}{\rho c} \frac{\partial^2 T}{\partial x^2} \ ,
\end{equation}
die die räumliche als auch zeitliche Entwicklung der Temperatur beschreibt. Aus der Formel \ref{eqn:KG} lässt sich eine Temperaturleitfähigkeit von
\begin{equation}
	\sigma T = \frac{\kappa}{\rho c}
\end{equation}
ablesen. Dies ist ein Maß dafür wie schnell sich Temperaturunterschiede ausgleichen. Wenn ein Stab mit der Periode $T$ abwechselnd erwärmt und gekühlt wird kommt es zu einer erzwungenen Schwingung der Form 
\begin{equation}
	T(x,t) = T_{max}e^{-\sqrt{\frac{\omega \rho c}{2 \kappa}}x}cos \left( \omega t - \sqrt{ \frac{\omega \rho c}{2 \kappa}}x \right) \ .
\end{equation}
Durch ablesen der Koeffizienten der Wellengeleichung und dem Zusammenhang das die Phasengeschwindigkeit gleich der Kreisfrequenz durch die Wellenzahl ist erhält man eine Geschwindikeit von
\begin{equation}
	\label{eqn:v}
	v = \sqrt{ \frac{2 \kappa \omega}{\rho c}} \ .
\end{equation}
Den Dämpfungsfaktor erhält man aus dem Amplitudenverhältniss $A_{nah}$ und $A_{fern}$ sowie $\Delta x$
\begin{equation}
	\label{eqn:kappa}
	\kappa = \frac{\rho c (\Delta x)^2}{2 \Delta t \, ln(A_{nah}/A_{fern})}
\end{equation}

\subsection{Fehlerrechnung}
Sämtliche Fehlerrechnungen werden mit Hilfe von Python 3.4.3 durchgeführt.
\subsubsection{Mittelwert}
Der Mittelwert einer Messreihe $x_1, ... ,x_n$ lässt sich durch die Formel
\begin{equation}
	\label{eqn:Mittelwert}
	\overline{x} = \frac{1}{N} \sum_{k=1}^N x_k
\end{equation}
berechnen. Die Standardabweichung des Mittelwertes beträgt 
\begin{equation}
	\label{eqn:Mittelwert_err}
	\Delta \overline{x} = \sqrt{ \frac{1}{N(N-1)} \sum_{k=1}^N (x_k - \overline{x})^2}
\end{equation}

\subsubsection{Gauß'sche Fehlerfortpflanzung}
Wenn $x_1, ..., x_n$ fehlerbehaftete Messgrößen im weiteren Verlauf benutzt werden, wird der neue Fehler $\Delta f$ mit Hilfe der Gaußschen Fehlerfortpflanzung angegeben.
\begin{equation}
	\Delta f = \sqrt{\sum_{k=1}^N \left( \frac{ \partial f}{\partial x_k} \right) ^2 \cdot (\Delta x_k)^2} \ .
\end{equation}

\subsubsection{Lineare Regression}
Die Steigung und y-Achsenabschnitt einer Ausgleichsgeraden werden gegebenfalls mittels Linearen Regression berechnet. 
\begin{equation}
	y = m \cdot x + b
\end{equation}
\begin{equation}
	m = \frac{ \overline{xy} - \overline{x} \overline{y} } {\overline{x^2} - \overline{x}^2}
\end{equation}
\begin{equation}
	b = \frac{ \overline{x^2}\overline{y} - \overline{x} \, \overline{xy}} { \overline{x^2} - \overline{x}^2}
\end{equation}
