\section{Diskussion}
\label{sec:Diskussion}
\subsection{Aluminium und Messing}
Die Abweichungen der ermitteltete Wärmekapazität von Messing und der der Realität liegt bei 37,6 \% und die Abweichung bei Messing beträgt 28,1 \%. Mögliche Gründe für die Abweichung sind Ablesefehler bei der Auswertung des Graphens. Eine Fehlerreduzierung wäre durch eine elektronische Auswertung des Graphens möglich. Desweiteren waren die Stäbe nicht optimal von der Umgebung abgeschirmt und die Netzspannung des Netzgerätes im Laufe der Messung um 0.1 Volt geschwankt hat.

\subsection{Edelstahl}
Eine Auswertung des Graphens ist nicht möglich, da keine Amplituden beim weiter entferneten Temperaturverlauf des Termoelementes mehr erkennen zu sind. Ursache dafür könnte eine zu geringe Wärmeleitfähigkeit seien. Auffällig ist zudem auch, dass die Skallierung der Zeitachse nicht mit dem Graphen übereinstimmt.
