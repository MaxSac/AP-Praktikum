\section{Theoretische Grundlagen}
\label{sec:Theorie}
\subsection{Versuch}

Ziel des Versuches ist die Bestimmung der Trägheitsmomente von verschiedenen Objekten. Dazu führen die Versuchsobjekte eine Rotationsbewegung aus, welche durch das Drehmoment M, den Auslenkwinkel $\varphi$ und dem Trägheitsmoment $I$ charakterisiert ist.
\newline \newline
Das Trägheitsmoment ist ein Maß des Widerstandes einer Rotationsbewegung um die Drehachse. Im einfachsten Fall dreht sich ein Massepunkt mit dem Abstand r um die Drehachse. Dabei fallen Drehachse und Schwerpunkt des Systems auf die selbe Achse.
\begin{equation}
	I = \sum_i \vec{r_i^2}  \cdot  m_i
\end{equation}   
Falls jedoch das getestete Objekt ein Volumen besitzt muss über die einzelnen Masseelemente $dm$ integriert werden. Bei einer kontinuierlichen Masseverteilung kann auch analog dazu die Dichte $\varrho$ über das Volumen $dV$ integriert werden. 
\begin{equation}
	I = \int{ r^2 \, \symup{dm}}
\end{equation}
Wenn der Schwerpunkt von der Drehachse parallel um den Abstand a verschoben ist wird ein Korrekturfaktor nötig. Dieser folgt aus dem Steiner'schen Satz und berechnet sich aus der Verschiebung des Schwerpunktes um die Distanz a und der Masse des Objetktes. Das Gesamte Trägheitsmoment eines Objekt ist also die Summe aus dem Trägheitsmoment des Schwerpunktes ($I_s$) und des Steiner'schens Anteils.
\begin{equation}
	\label{eqn:Steiner}
	I = I_s + m  \cdot a^2
\end{equation} 
Die Bewegung des schwingungfähigen Objektes um den Winkel $\varphi$ mit dem Moment $\vec{M} = D \cdot  \varphi$ lässt sich durcht die  Bewegungsgleichung 
\begin{equation}
\ddot{\alpha}= - \frac{D}{I} \cdot \alpha
\end{equation}
beschreiben. Dieser entnimmt man eine Schwingungsdauer von 
\begin{equation}
	\label{eqn:Schwingungsdauer}
	T = 2 \pi \sqrt{\frac{I}{D}} 
\end{equation}

\subsection{Fehlerrechnung}
\label{sec:Fehler}
Sämtliche Fehlerrechnungen werden mit Hilfe von Python 3.4.3 durchgeführt.
\subsubsection{Mittelwert}
Der Mittelwert einer Messreihe $x_1, ... ,x_n$ lässt sich durch die Formel
\begin{equation}
        \overline{x} = \frac{1}{N} \sum_{k=1}^N x_k
\end{equation}
berechnen. Die Standardabweichung des Mittelwertes beträgt
\begin{equation}
        \Delta \overline{x} = \sqrt{ \frac{1}{N(N-1)} \sum_{k=1}^N (x_k - \overline{x})^2}
\end{equation}

\subsubsection{Gauß'sche Fehlerfortpflanzung}
Wenn $x_1, ..., x_n$ fehlerbehaftete Messgrößen im weiteren Verlauf benutzt werden, wird der neue Fehler $\Delta f$ mit Hilfe der Gaußschen Fehlerfortpflanzung angegeben.
\begin{equation}
        \label{eqn:Gauß}
	\Delta f = \sqrt{\sum_{k=1}^N \left( \frac{ \partial f}{\partial x_k} \right) ^2 \cdot (\Delta x_k)^2}
\end{equation}

\subsubsection{Lineare Regression}
Die Steigung und y-Achsenabschnitt einer Ausgleichsgeraden werden gegebenfalls mittels Linearen Regression berechnet.
\begin{equation}
        y = m \cdot x + b
\end{equation}
\begin{equation}
	\label{eqn:m}
        m = \frac{ \overline{xy} - \overline{x} \overline{y} } {\overline{x^2} - \overline{x}^2}
\end{equation}
\begin{equation}
	\label{eqn:b}
        b = \frac{ \overline{x^2}\overline{y} - \overline{x} \, \overline{xy}} { \overline{x^2} - \overline{x}^2}
\end{equation}


