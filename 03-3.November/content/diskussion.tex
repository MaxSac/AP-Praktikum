\section{Diskussion}
\label{sec:Diskussion}
\subsection{Drillachse}
Aufgrund der großen Messunsicherheit der Zeit bei der Bestimmung der Drillachse wird ein negatives Trägheitsmoment ermittelt. Dies hatt keinen erklärlichen physikalischen Sinn und wird deswegen im weiteren Verlauf vernachlässigt.

\subsection{Verschiedene Objekte}
Die Abweichung der praktisch ermittelten Werte von den theoretischen beträgt beim Zylinder  2.6 \% und bei der Kugel 5.9 \%. Diese Abweichung können durch die ungenaue Zeitmessung erklärt werden und auf die Messfehler bei der Volumen Bestimmung.

\subsection{Puppe}
Der Fehler bei der Messung des Trägheitsmomentes der Puppe mit den ausgestreckten Armen beträgt 26.8 \%. Dies lässt sich durch die geringe Periodendauer der Schwingung erklären. Außerdem werden die Körperteile der Puppe als Zylinder genähert wobei große unterschiede zwischen dem theoretischem Wert und der Realität auftreten können. Zusätzlich verändert die Puppe ihre Haltung während sie schwingt.
\ \\
Bei der Puppe mit den angelegten Armen kommt zusätzlich das Problem hinzu, dass das Trägheitsmoment zu klein wird, wobei das Trägheitsmoment der Drillachse berücksichtigt werden müsste. Dies ist jedoch mit dem Aufbau nicht genau genug bestimmbar. Der Fehler zwischen dem theoretischem und gemessenen Wert liegt in einer Größenordnung.
