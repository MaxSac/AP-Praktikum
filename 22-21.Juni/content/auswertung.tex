\section{Auswertung}
\label{sec:Auswertung}
\subsection{\texorpdfstring{$\gamma$ -Strahlung }%
                               {gamma -Strahlung}}
Die Nullmessung ohne radiaktives Präperat ergibt eine Hintergrundstrahlung von 
\begin{equation}
  (\num{800 +- 30}) \text{ counts in } 900 \text{s} \ .
  \label{eqn:gammacoun}
\end{equation}
Normiert man diese Impulszahl auf die Zählrate ergibt sich die Nullaktivität $A_0$ von
\begin{equation}
  A_0 = (\num{0.89 +- 0.03}) \, \text{cps} \ .
  \label{eqn:A0}
\end{equation}
Im weiteren Verlauf werden von den gemessenene Messwerten die Zählraten der Hintergrund abgezogen. Die gemessenen Impulsraten in Abhängigkeit der Abschirmdicke für Blei, sind in Tabelle \ref{tab:ABlei} aufgetragen, sowie die bereinigte Zählrate.
\begin{table}
  \centering
  \begin{tabular}{c c c c}
    \toprule
    Dicke d / $10^{-3}$ m & Zeit t / s & Zählrate N & Aktivität A / $\frac{1}{s}$ \\
    \midrule
    	1.2	&40 	&5276	&\num{130 +- 10} 	\\
	2.4	&40 	&4776	&\num{120 +- 10}	\\
	3.6	&40 	&4259	&\num{100 +- 10}	\\
	10.0	&40 	&1907	&\num{48 +- 7}	\\
	11.2	&40 	&1811	&\num{45 +- 7}	\\
	12.4	&40 	&1644	&\num{41 +- 6}	\\
	13.6	&40 	&1479	&\num{37 +- 6}	\\
	20.0	&40 	&707	&\num{18 +- 4}	\\
	21.2	&40 	&624	&\num{16 +- 4}	\\
	22.4	&40 	&598	&\num{15 +- 4}	\\
	23.4	&40 	&543	&\num{14 +- 4}	\\
	30.0	&40 	&280	&\num{7 +- 3}	\\
	40.0	&100	&292	&\num{3 +- 2}	\\
	50.0	&200	&294	&\num{1 +- 1}	\\
	60.0	&250	&304	&\num{1 +- 1}	\\
    \bottomrule
  \end{tabular}
  \caption{Aktivität nach Abschirmung durch Blei}
  \label{tab:ABlei}
\end{table}
In Diagramm \ref{fig:Blei} wird der Massenbelegung gegen die auf die logarithmierte y-Achse Zählrate aufgetragen. Anhand eines Fittes der Form 
\begin{equation}
  y_\text{fit} = \text{exp}(a \cdot x + b) 
  \label{eqn:fit}
\end{equation} 
wird ein Absorbtionskoeffizient µ von 
\begin{equation}
  \mu_\text{Blei} = (\num{106 +- 2}) \frac{1}{m}
  \label{eqn:absK}
\end{equation}
ermittelt so wie eine Anfangsaktivität $N(0)$ von 
\begin{equation}
  N_\text{Blei}(0) = \num{148 +- 6} \ .
  \label{eqn:NBlei}
\end{equation}
\begin{figure}
  \centering
  \includegraphics[height=8cm]{Blei.pdf}
  \caption{Zählrate in Abhängigkeit der Abschirmung durch Blei}
  \label{fig:Blei}
\end{figure}
Der mittels Formel \ref{eqn:} und \ref{eqn:} berechnet Compton-Absorptionskoeffizient beträgt
\begin{equation}
  \mu_\text{com} = 68.14 \, \frac{1}{\text{m}} \ .
  \label{eqn:}
\end{equation}
Bei der berechnung wurde eine Molvolumen von 18.26 $\text{cm}^3 / \text{mol}^{-1}$ (\cite{Kupfer}) angenommen.
Die gemessenen Werte für Kupfer und die Zählrate sind in Tabelle \ref{tab:AKupfer} aufgetragen. Der Absorbtionskoeffizient sowie die Anfangsaktivität wird analog zum Blei aus den Fitts in Abbildung \ref{fig:Kupfer} berechnet. 
\begin{table}
  \centering
  \begin{tabular}{c c c c}
    \toprule
    Dicke d / $10^{-3}$ m & Zeit t / s & Zählrate N & Aktivität A / $\frac{1}{s}$ \\
    \midrule
    0.5		&40	&5484&	\num{140 +- 10} \\	
    1.0		&40	&5459&	\num{140 +- 10} \\	
    1.5		&40	&5314&	\num{130 +- 10} \\	
    2.0		&40	&4413&	\num{110 +- 10} \\	
    2.5		&40	&5168&	\num{130 +- 10} \\	
    3.1		&40	&4982&	\num{120 +- 10} \\	
    3.6		&40	&4682&	\num{120 +- 10} \\	
    4.1		&40	&4351&	\num{110 +- 10} \\	
    5.0		&40	&4498&	\num{110 +- 10} \\	
    5.5		&40	&4165&	\num{100 +- 10} \\	
    6.0		&40	&4383&	\num{110 +- 10} \\	
    6.5		&40	&4192&	\num{100 +- 10} \\	
    7.0		&40	&4444&	\num{110 +- 10} \\	
    8.1		&40	&4169&	\num{100 +- 10} \\	
    10.0	&40	&3475&	\num{90 +- 10} \\	
    13.1	&40 	&2784&	\num{70 +- 9} \\   	  
    18.1	&40	&2183&	\num{55 +- 8} \\	
    20.0	&40	&2246&	\num{56 +- 7} \\	
    30.0	&40 	&1319&	\num{34 +- 6} \\	     
    40.0    	&40 	&719&	\num{18 +- 4} \\	 
    50.0    	&40 	&601&	\num{15 +- 4} \\	 
    60.0	&40 	&328&	\num{8 +- 3} \\	 
    \bottomrule
  \end{tabular}
  \caption{Aktivität nach Abschirmung durch Kupfer}
  \label{tab:AKupfer}
\end{table}
Die Aktivität von Kupfer beträgt
\begin{equation}
  \mu_\text{Kupfer} = \num{ 47 +- 3 } \, \frac{1}{m}
  \label{eqn:muku}
\end{equation}
und die Anfangsaktivität bestimmt über die Kupferabschirmung von
\begin{equation}
  N_\text{Kupfer}(0) = \num{134 +- 13} \, \frac{1}{s} \ .
  \label{eqn:NKu}
\end{equation}
\begin{figure}
  \centering
  \includegraphics[height=8cm]{Kupfer.pdf}
  \caption{Zählrate in Abhängigkeit der Abschirmung durch Kupfer}
  \label{fig:Kupfer}
\end{figure}
Der unter verwendung des Molvolumens von 7.11 (\cite{Kupfer}) berechnete theoretische Compton-Absorptionskoeffizient beträgt 
\begin{equation}
  \nu_\text{comp} = 61.89 \frac{1}{m} \ .
  \label{eqn:Kc}
\end{equation}
\subsection{\texorpdfstring{$\beta$ -Strahlung }%
                               {beta -Strahlung}}
Die Hintergrundstrahlung des $\beta$-Strahlungsversuch beträgt 
\begin{equation}
  (\num{248 +- 15}) \text{ cps in } 900 \text{s}
  \label{eqn:hintb}
\end{equation}
woraus sich eine Hintergrundaktivität von 
\begin{equation}
  (\num{0.275 +- 0.004}) \frac{1}{s}
  \label{eqn:HAB}
\end{equation} 
berechnet. Diese wird im weiteren Verlauf von den berechneten Aktivitäten abgezogen. Die gemessenen Zählraten zu den enstprechenden Dicken, sowie die berechneten Aktivitäten sind in Tabelle \ref{tab:AAlu} aufgelistet. 
\begin{table}
  \centering
  \begin{tabular}{c c c c}
    \toprule
    Dicke d / $10^{-6}$ m & Zeit t / s & Zählrate N & Aktivität A / $\frac{1}{s}$ \\
    \midrule
    102&	40 &	1374&	\num{34 +-6 } \\	
    126&	40 &	719&	\num{18 +-4 } \\	
    153&	40 &	297&	\num{7 +-3 } \\	
    160&	80 &	360&	\num{5 +-2 } \\	
    200&	150&	275&	\num{2 +-1 } \\	
    253&	450&	268&	\num{0.6 +-0.8 } \\	
    302&	600&	319&	\num{0.5 +-0.7 } \\	
    338&	650&	336&	\num{0.5 +-0.7 } \\	
    400&	650&	355&	\num{0.5 +-0.7 } \\	
    444&	600&	315&	\num{0.5 +-0.7 } \\	
    482&	600&	340&	\num{0.6 +-0.8 } \\	
    \bottomrule
  \end{tabular}
  \caption{Aktivität nach Abschirmung durch Aluminium}
  \label{tab:AAlu}
\end{table}
In dem Diagramm \ref{fig:Alu} wird die Massenbelgung gegen die Zählrate aufgetragen. 
\begin{figure}
  \centering
  \includegraphics[height=8cm]{Aluminium.pdf}
  \caption{Ausgleichungsrechnung zur Bestimmung der Gesamtenergie}
  \label{fig:Alu}
\end{figure}
Mittels dem Schnittpunkt von zwei Linearen Fitts der From \eqref{eqn:fit} wird die Maximale Massenbelgeung berechnet. Sie beträgt 
\begin{equation}
  R_\text{Max} = \num{0.65 +- 0.07} \, \frac{\text{kg}}{\text{m}^2} \ . 
  \label{eqn:Rmax}
\end{equation}
Aus der Maximalen Massenbelgung wird die freiwerdende Gesamtenergie mittels Formel \eqref{eqn:} berechnet und beträgt
\begin{equation}
  E_\text{max} = \num{1.4 +- 0.1} \, \text{MeV}
  \label{}
\end{equation}
berechnet.
