\section{Auswertung}
\label{sec:Auswertung}

Der zu Beginn ausgemessene Abstand $L$ von dem Spalt zu der Messdiode und die Wellenlänge des verwendeten Lasers lauten:
\begin{align*}
  L &= 0.93 \ \text{m} \\
  \lambda &= 633 \cdot 10^{-9} \ \text{m}
\end{align*}
Alle Messwerte zur Bestimmung der Spaltgrößen $b$, über das Interferenzmuster, befinden sich in Tabelle \ref{tab:b1} bis \ref{tab:b4}. Der gemessene Dunkelstrom beträgt,
\begin{align*}
  I_\text{D} = 0.1 \cdot 10^{-9} \ \text{A}
\end{align*}
und wird für alle Messungen als konstant angesehen. Mithilfe der Messwerte und Gleichung (\ref{eqn:I}) für die Einzelspalte und mit Gleichung (\ref{eqn:Id}) für den Doppelspalt wird eine Ausgleichsrechnung durchgeführt. Der Winkel wird als $\Phi \approx \frac{x}{L}$ angenommen. Da in beiden Gleichungen durch $\sin(\frac{x}{L})$ geteilt wird, muss der Wert $x = 0$ aus der Ausgleichsrechnung entfernt werden.

\begin{table}[H]
  \centering
  \begin{tabular}{c c}
    \toprule
    a / mm & I / m A \\
    \midrule
  -25	& 0.00062 \\
  -24	& 0.00058 \\
  -23	& 0.00046 \\
  -22	& 0.00032 \\
  -21	& 0.00028 \\
  -20	& 0.00038 \\
  -19	& 0.00092 \\
  -18	& 0.00130 \\
  -17	& 0.00180 \\
  -16	& 0.00225 \\
  -15	& 0.00225 \\
  -14	& 0.00175 \\
  -13	& 0.00110 \\
  -12	& 0.00060 \\
  -11	& 0.00075 \\
  -10	& 0.00200 \\
  -9	& 0.00600 \\
  -8	& 0.01250 \\
  -7	& 0.02200 \\
  -6	& 0.03600 \\
  -5	& 0.05400 \\
  -4	& 0.07200 \\
  -3	& 0.08800 \\
  -2	& 0.09000 \\
  -1	& 0.10000 \\
  0	  & 0.12500 \\
  1	  & 0.10000 \\
  2	  & 0.10000 \\
  3	  & 0.09000 \\
  4	  & 0.07400 \\
  5	  & 0.05600 \\
  6	  & 0.04000 \\
  7	  & 0.02400 \\
  8	  & 0.01200 \\
  9	  & 0.00750 \\
  10	& 0.00300 \\
  11	& 0.00150 \\
  12	& 0.00140 \\
  13	& 0.00225 \\
  14	& 0.00260 \\
  15	& 0.00340 \\
  16	& 0.00340 \\
  17	& 0.00280 \\
  18	& 0.00200 \\
  19	& 0.00175 \\
  20	& 0.00125 \\
  21	& 0.00100 \\
  22	& 0.00075 \\
  23	& 0.00090 \\
  24	& 0.00100 \\
  25	& 0.00100 \\
    \bottomrule
  \end{tabular}
  \caption{}
  \label{tab:}
\end{table}
