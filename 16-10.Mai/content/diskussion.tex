\section{Diskussion}
\label{sec:Diskussion}
Die aus dem Versuch relevanten erechnete Messwerte sind in Tabelle \ref{tab:sum} aufgeführt.
\begin{table}[H]
  \centering
  \begin{tabular}{c|c c}
    \toprule
	\multirow{2}{*}{Zu Messende Größe} & \multirow{2}{*}{Ermittelter Wert} & Relative Abweichung \\
	& & vom Theoriewert \\
    \midrule
     spezifische Ladung & $(\num{1.77 +- 0.02})10^{11}$ C/kg & 1 \%	\\
     Erdmagnetfeld 	  &  4.34 \cdot $10^{-5}$ T       & 11 \% 	\\
     Kenngröße der Braunsche Röhre& $(\num{0.27 +- 0.01}) \frac{1}{m}$ & 19 \% \\
     Frequenz der Sinusspannung& $(\num{79.81 +- 0.05})$ Hz & 0.2 \% \\
     Scheitelwert der Sinusspannung& $ (\num{6.0 +- 0.2}) V $& / \\
    \bottomrule
  \end{tabular}
  \caption{Messergebnisse mit realtiven Abweichung}
  \label{tab:sum}
\end{table}
Die Messung der Spezifische Ladung kann als gelungen gewertet werden. Desweiteren liegen alle Messwerte ziemlich nah an der Ausgleichsgraden. Eine mögliche Ursache für die Messabweichung bei der Bestimmung des Erdmagnetfeld ist die geringe Anzahl an Messwerte, sowie das ungenaue justieren des Gegenfeldes. Es erweist sich als schwierig einen genauen Punkt auf dem Schirm zu definieren, da dieser sich je nach Blickwinkel verschiebt. Die Empfindlichkeit der Braunschen Röhre kann nicht genau mit dem Theoriewert verglichen werden, da die Formel nicht der Geometrie der Ablenkkondensatoren entspricht. Sie wurden zum Vergleich als Plattenkondensatoren genähert. Hier liegen die Messerte ebenfalls nah am Fit. Die gemittelte Sägezahnfrequenz entspricht ziemlich genau der Theoretischen und über die Scheitelwerte lässt sich keine Aussage treffen. 
