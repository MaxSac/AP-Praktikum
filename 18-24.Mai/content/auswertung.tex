\section{Auswertung}
\label{sec:Auswertung}
Für die weitere Versuchsauswertung wird zunächst die mittelere freie Wellenlänge für die verwendeten Temperaturen nach Gleichung \ref{eqn:} und \ref{eqn:} ausgerechnet. Die freien Wellenlängen und das Verhältnis des Abstandes zwischen Kathode-Beschleunigungselektrode $a$ zur freien Wellenlänge $w$ sind in Tabelle \ref{tab:mitWel} aufgeführt. 
\begin{table}
  \centering
  \begin{tabular}{c c c}
    \toprule
    	Temperatur $T$ / °C & mittelere Wellenlänge $w$ / m & Verhältnis $\frac{a}{w}$ \\
    \midrule
    	27  & $4.74 \cdot 10^{-3}$	& 2			\\
      	105 & $4.19 \cdot 10^{-5}$	& 2 $\cdot 10^2$	\\
	140 & $8.96 \cdot 10^{-6}$	& 1 $\cdot 10^3$	\\
	180 & $2.06 \cdot 10^{-6}$	& 5 $\cdot 10^3$	\\
	190 & $1.49 \cdot 10^{-6}$	& 6 $\cdot 10^3$	\\
	\bottomrule
  \end{tabular}
  \caption{mittlere Wellenlänge und Verhältniss zur Beschleunigungsstrecke}
  \label{tab:mitWel}
\end{table}

\subsection{Differentielle Energieverteilung}
Als erstes wird die Skalierung der x-Achse für die Graphen bei 27 und 140 °C bestimmt. Der Graph welcher bei 27 °C aufgezeichnet wurde, weist eine Distanz von 21 cm auf zwischen dem x-Achsen Punkt von 0 bis 10 Volt auf. Somit ergibt sich für die Länge eines cm Kästchen $K$ eine Spannung von 
\begin{equation}
  K1_\text{cm} = 0.476 V . 
  \label{eqn:K1}
\end{equation}
Für den Graphen bei 140 °C wird eine Stecke von 20.7 cm ausgemessen und ein Kästchenmaßstab von 
\begin{equation}
  K2_\text{cm} = 0.483 V
  \label{eqn:K2}
\end{equation}
berechnet. 
In die Graphen wird die Steigung der Funktion mittels Steigungsdreieck approximiert. Die Steigungen zu den verschiedenen Spannungen sind in Tabelle \ref{tab:steig} aufgeführt.
\begin{table}
  \centering
  \begin{tabular}{c c|c c}
    \toprule
    	\multicolumn{2}{c}{$T = 27$} & \multicolumn{2}{c}{$t = 140$} \\
	$x_1 \cdot K1$ / V & $\Delta y_1  / \Delta x_1$ & $x_2 \cdot K2$ / V & $\Delta y_2  / \Delta x_2$ \\
    \midrule
	0.07 	& 0.0 	& 0.07 	& 1.7	\\
	0.38	& 0.1	& 0.38 	& 1.2	\\
	0.86	& 0.1	& 0.87	& 1.3	\\
	1.33	& 0.1	& 1.35	& 1.4	\\
	1.81	& 0.1	& 1.84	& 1.4	\\
	2.28	& 0.1	& 2.32	& 1.5	\\
	2.76	& 0.1	& 2.80 	& 1.5 	\\
	3.23	& 0.1	& 3.28	& 1.4	\\
	3.71	& 0.1	& 3.77	& 1.4	\\
	4.19	& 0.2	& 4.25	& 1.0	\\
      	4.66	& 0.2	& 4.61	& 1.0	\\
	5.14	& 0.2	& 4.85	& 0.4 	\\
	5.62	& 0.2	& 5.22 	& 0.2	\\
	6.10	& 0.3	& 5.70	& 0.2	\\
	6.57	& 0.3	& 6.18	& 0.2	\\
	6.81	& 0.4	& 6.67	& 0.1	\\
	7.04	& 0.4	& 7.15	& 0.2	\\
	7.28	& 0.4	& 7.63	& 0.2 	\\
	7.52	& 0.4	& 8.11	& 0.2	\\
	7.76	& 0.6	& 8.59	& 0.1	\\
	8.00	& 0.8	& 9.08	& 0.1	\\
	8.23	& 0.8	& 9.56	& 0.1	\\
	8.47	& 1.0	& 9.90	& 0.0	\\
	8.71	& 1.6	& & \\
	8.94	& 1.8	& & \\
	9.18	& 3.6	& & \\
	9.42	& 9.0	& & \\
	9.66	& 4.4	& & \\
	9.90 	& 0.8	& & \\
	10.14 	& 0.0	& & \\
    \bottomrule
  \end{tabular}
  \caption{Steigung der Graphen in Abhängigkeit der Bremsspannung}
  \label{tab:steig}
\end{table}
Die Steigungen werden in den Diagrammen \ref{fig:Energie20} und \ref{fig:} gegen die Bremsspannung aufgetragen. Dem Graphen \ref{fig:Energie20} kann entnommen werden, dass der größte Teil der Elektronen eine Energie von 9.42 eV besitzen.
\begin{figure}
  \centering
  \includegraphics[height=8cm]{20Grad.pdf}
  \caption{<+caption text+>}
  \label{fig:Energie20}
\end{figure}

\begin{figure}
  \centering
  \includegraphics[height=8cm]{140Grad.pdf}
  \caption{<+caption text+>}
  \label{fig:<+label+>}
\end{figure}
