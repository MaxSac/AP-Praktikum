\section{Auswertung}
\label{sec:Auswertung}
Für die weitere Versuchsauswertung wird zunächst die mittelere freie Wellenlänge für die verwendeten Temperaturen nach Gleichung \ref{eqn:} und \ref{eqn:} ausgerechnet. Die freien Wellenlängen und das Verhältnis des Abstandes zwischen Kathode-Beschleunigungselektrode $a$ zur freien Wellenlänge $w$ sind in Tabelle \ref{tab:mitWel} aufgeführt. 
\begin{table}
  \centering
  \begin{tabular}{c c c}
    \toprule
    	Temperatur $T$ / °C & mittelere Wellenlänge $w$ / m & Verhältnis $\frac{a}{w}$ \\
    \midrule
    	27  & $4.74 \cdot 10^{-3}$	& 2			\\
      	105 & $4.19 \cdot 10^{-5}$	& 2 $\cdot 10^2$	\\
	140 & $8.96 \cdot 10^{-6}$	& 1 $\cdot 10^3$	\\
	180 & $2.06 \cdot 10^{-6}$	& 5 $\cdot 10^3$	\\
	190 & $1.49 \cdot 10^{-6}$	& 6 $\cdot 10^3$	\\
	\bottomrule
  \end{tabular}
  \caption{mittlere Wellenlänge und Verhältniss zur Beschleunigungsstrecke}
  \label{tab:mitWel}
\end{table}

\subsection{Differentielle Energieverteilung}
Als erstes wird die Skalierung der x-Achse für die Graphen bei 27 und 140 °C bestimmt. Der Graph welcher bei 27 °C aufgezeichnet wurde, weist eine Distanz von 21 cm auf zwischen dem x-Achsen Punkt von 0 bis 10 Volt auf. Somit ergibt sich für die Länge eines cm Kästchen $K$ eine Spannung von 
\begin{equation}
  K1_\text{cm} = 0.476 V . 
  \label{eqn:K1}
\end{equation}
Für den Graphen bei 140 °C wird eine Stecke von 20.7 cm ausgemessen und ein Kästchenmaßstab von 
\begin{equation}
  K2_\text{cm} = 0.483 V
  \label{eqn:K2}
\end{equation}
berechnet. 
In die Graphen wird die Steigung der Funktion mittels Steigungsdreieck aproximiert. Die Steigungen zu den verschiedenen Spannungen sind in Tabelle \ref{tab:steig} aufgeführt.
\begin{table}
  \centering
  \begin{tabular}{c c|c c}
    \toprule
    	\multicolumn{2}{c}{$T = 27$} & \multicolumn{2}{c}{$t = 140$} \\
	$x_1 \cdot K1$ / V & $\Delta y_1  / \Delta x_1$ & $x_2 \cdot K2$ / V & $\Delta y_2  / \Delta x_2$ \\
    \midrule
	0.07 	& 0.0 	& 0.07 	& 1.7	\\
	0.38	& 0.1	& 0.38 	& 1.2	\\
	0.86	& 0.1	& 0.87	& 1.3	\\
	1.33	& 0.1	& 1.35	& 1.4	\\
	1.81	& 0.1	& 1.84	& 1.4	\\
	2.28	& 0.1	& 2.32	& 1.5	\\
	2.76	& 0.1	& 2.80 	& 1.5 	\\
	3.23	& 0.1	& 3.28	& 1.4	\\
	3.71	& 0.1	& 3.77	& 1.4	\\
	4.19	& 0.2	& 4.25	& 1.0	\\
      	4.66	& 0.2	& 4.61	& 1.0	\\
	5.14	& 0.2	& 4.85	& 0.4 	\\
	5.62	& 0.2	& 5.22 	& 0.2	\\
	6.10	& 0.3	& 5.70	& 0.2	\\
	6.57	& 0.3	& 6.18	& 0.2	\\
	6.81	& 0.4	& 6.67	& 0.1	\\
	7.04	& 0.4	& 7.15	& 0.2	\\
	7.28	& 0.4	& 7.63	& 0.2 	\\
	7.52	& 0.4	& 8.11	& 0.2	\\
	7.76	& 0.6	& 8.59	& 0.1	\\
	8.00	& 0.8	& 9.08	& 0.1	\\
	8.23	& 0.8	& 9.56	& 0.1	\\
	8.47	& 1.0	& 9.90	& 0.0	\\
	8.71	& 1.6	& & \\
	8.94	& 1.8	& & \\
	9.18	& 3.6	& & \\
	9.42	& 9.0	& & \\
	9.66	& 4.4	& & \\
	9.90 	& 0.8	& & \\
	10.14 	& 0.0	& & \\
    \bottomrule
  \end{tabular}
  \caption{Steigung der Graphen in Abhängigkeit der Bremsspannung}
  \label{tab:steig}
\end{table}
Die Steigungen werden in den Diagrammen \ref{fig:Energie20} und \ref{fig:Energie140} gegen die Bremsspannung aufgetragen. Dem Graphen \ref{fig:Energie20} kann entnommen werden, dass der größte Teil der Elektronen eine Energie von 9.42 eV besitzen.
\begin{figure}
  \centering
  \includegraphics[height=8cm]{20Grad.pdf}
  \caption{Differentielle Energieverteilung 27 °C}
  \label{fig:Energie20}
\end{figure}
Die Beschleunigungsspannung beträgt beträgt 11 V, woraus ein Kontaktpotential K von 1.58 V resultiert.
\begin{figure}
  \centering
  \includegraphics[height=8cm]{140Grad.pdf}
  \caption{Differentielle Energieverteilung 140 °C}
  \label{fig:Energie140}
\end{figure}
Desweiteren weisen die Peaks in beiden Diagrammen eine gewisse Unschärfe auf, was an der Fermi-Dirac-Statistik liegt. Zur Abbildung \ref{fig:Energie140} ist zu sagen das der Graph eine Steigung aufweist. Dies lässt sich durch das Verhältniss $a/w$ erklären, da aufgrund der kleinen mittleren freien Wellenlänge es zu vielen elastischen Stößen kommt. Dabei werden die Elektronenen um eine zufällige Geschwindigkeit abgebremst und die Geschwindigkeit in Richtung Auffängerelektrode kann nicht mehr bestimmt werden. Aufgrund dessen verteilen sich die Energien ebenfalls zufällig. Ab einer Energie von 4.9eV ist ein Einbruch in der Energieverteilung zu erkennen. Dieser lässt sich aufgrund der Anregung der Queksilberatome erklären. Dieses Phänomen tritt ab einer Spannung von 4.76 V auf, worauf in den folgenden Kapitel noch weiter eingegangen wird.

\subsection{Frank-Hertz Kurve}
Analog zum vorhergegangenen Auswertungsteil wird zunächst der Kästchenmaßstab bestimmt. Dieser beträgt 
\begin{equation}
  K_3 = 2.43 V
  \label{eqn:K3}
\end{equation}
beträgt. Die Distanzen zwischen den Maxima und dem Darauffolgenden werden mit dem Kästchenmaßstab $K2$ multipliziert und in Tabelle \ref{tab:Max} aufgetragen.
\begin{table}
  \centering
  \begin{tabular}{c c}
    \toprule
    	k & $(U_{k+1} - U_k)$ / V \\
    \midrule
    	1 & 4.62	\\
    	2 & 4.37	\\
    	3 & 4.62	\\
    	4 & 4.62	\\
    	5 & 4.86	\\
    	6 & 4.86	\\
    	7 & 4.86	\\
    	8 & 5.34	\\
    \midrule
    	$\overline{U}$  = & 4.76 \\
    \bottomrule
  \end{tabular}
  \caption{Abstand zu dem darauffolgendem Maxima}
  \label{tab:Max}
\end{table}
Aus dem gemittelten Anregungsenergie ergibt sich nach umstellen der Formel \ref{eqn:} eine Wellenlänge des Lichtes von 
\begin{equation}
  \lambda = \frac{c \cdot h \cdot e_0}{\overline{U}} = 259 \, \text{nm} \ .
  \label{eqn:spek}
\end{equation}
Somit liegt das Spektrum im Ultravioletten bereich und ist nicht für das Auge Sichtbar. Das erste Maximum wird bei einer Spannung von $U_{1 \text{Max}}$ = 6.80 V abgelesen. Aus Gleichung \ref{eqn:} folgt ein Kontaktpotential 
\begin{equation}
  K = 2.04 \, \text{V} \ .
  \label{eqn:KonFr}
\end{equation}
\subsection{Ionisierungsenergie}
Zur Bestimmung der Ionisierungsenergie wird Asymptote ins Diagramm gezeichnet und aus deren Schnittpunkt mit der Achse die Ionisierungsenergie abgelesen. Die Temperatur zum Zeitpunkt der Versuchsdurchführung im evakuierten Gefäß beträgt 105 °C. In dem Diagramm entspricht ein Kästchen 1 V und es wird ein Schnittpunkt $S$ von 11.8 cm abgelesen. Bereinigt man diesen Wert noch um das zuvor gemittelte Kontaktpotential ergibt sich eine Ionisierungsenergie von 
\begin{equation}
  E_\text{Io} = (S - \overline{K}) \cdot e_0 = 9.99 \, \text{eV} \ .
  \label{eqn:EIo}
\end{equation}
