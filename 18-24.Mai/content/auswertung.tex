\section{Auswertung}
\label{sec:Auswertung}
Für die weitere Versuchsauswertung wird zunächst die mittelere freie Wellenlänge für die verwendeten Temperaturen nach Gleichung \ref{eqn:} und \ref{eqn:} ausgerechnet. Die freien Wellenlängen und das Verhältnis des Abstandes zwischen Kathode-Beschleunigungselektrode $a$ zur freien Wellenlänge $w$ sind in Tabelle \ref{tab:} aufgeführt. 
\begin{table}
  \centering
  \begin{tabular}{c c c}
    \toprule
    	Temperatur $T$ / °C & mittelere Wellenlänge $w$ / m & Verhältnis \frac{a}{w} \\
    \midrule
    	27  & $4.74 \cdot 10^{-3}$	& 2			\\
      	105 & $4.19 \cdot 10^{-5}$	& 2 \cdot $10^2$	\\
	140 & $8.96 \cdot 10^{-6}$	& 1 \cdot $10^3$	\\
	180 & $2.06 \cdot 10^{-6}$	& 5 \cdot $10^3$	\\
	190 & $1.49 \cdot 10^{-6}$	& 6 \cdot $10^3$	\\
  \end{tabular}
  \caption{<+Caption text+>}
  \label{tab:<+label+>}
\end{table}<++>
