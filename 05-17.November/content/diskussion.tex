\section{Diskussion}
\label{sec:Diskussion}
Da alle Messgrößen fehlerbehaftet sind kommt es zur Fehlerfortpflanzung und daraus resultiert, dass auch die davon abgeleiteten Größen fehlerbehaftet sind. Zusätzlich ist der Versuchsaufbau nicht ideal isoliert und somit kommt es zum Wärmeaustausch mit der Umgebung. Desweiteren wird die Reibung des Gases an der Leitungswand als auch durch die Propeller erzeugte vernachlässigt. Im Fehler der Dichte des Wassers wird zwar berücksichtigt, dass das Wasser mit der Änderung der Temperatur seine Dichte ändert, jedoch ändert sich der Mittelwert des Wasserdichte nicht mit. Die Abbdichtung der Reservoire sind nicht ideal, da sie nicht optimal an dem Versuchsaufbau anschließen. Somit kann Wasser außerhalb des Kreislaufes verdunsten und Wärme mit der Umgebung ausgetauscht werden.


\section{Literaturverzeichnis}
\label{sec:Literaturverzeichnis}

[1] : TU Dortmund. Versuchsanleitung zum Experiment V206 - Die Wärmepumpe. 2015
