\section{Auswertung}
\label{sec:Auswertung}
\section{Druckkurve}
Dem Versuchsaufbau wurden die Temperatur T1 und T2, der Druck $p_{\text{b}}$ und $p_{\text{a}}$ sowie die Leistung des Kompressors entnommen. Die Messdaten werden in Tabelle ?? aufgelistet.
\begin{table}
  \centering
  \begin{tabular}{c c c c c c}
    \toprule
    $t$ / s & $T_{\text{1}}$ / K & $p_{\text{b}}$ /bar & $T_{2}$ /K & $P_{\text{a}}$ /bar & 
    Leistung /W \\
    \midrule
    0 	& 294.1 & 3	&4 &5&6 \\
    1 	& 294.7 &	&	\\
    2 	& 295.9 &	&	\\
    3 	& 296.9 &	&	\\
    4	& 298.2	&	&	\\
    5	& 299.4 &	&	\\
    6	& 300.7 &	&	\\
    7	& 302.0 &	&	\\
    8	& 303.2	&	&	\\
    9	& 304.4 &	&	\\
    10	& 305.5 &	&	\\
    11 	& 306.6 &	&	\\
    12	& 307.6	&	&	\\	
    13	& 308.7 &	&	\\	
    14 	& 309.7 &	&	\\
    15 	& 310.7 &	&	\\
    16 	& 311.6	&	&	\\
    17	& 312.5	&	&	\\
    18	& 313.5	&	&	\\
    19	& 314.3	&	&	\\
    20	& 315.2	&	&	\\
    21	& 316.0	&	&	\\
    22	& 316.8	&	&	\\
    23	& 317.5	&	&	\\
    24	& 318.3	&	&	\\
    25	& 319.0	&	&	\\
    26	& 319.8	&	&	\\
    27	& 320.5	&	&	\\
    28	& 321.2	&	&	\\
    29	& 321.8	&	&	\\
    30	& 322.5	&	&	\\
    31	& 323.3	&	&	\\
  \end{tabular}
  \caption{Dem Versuchsaufbau entommene Messgrößen}
  \label{tab:Daten}
\end{table}
